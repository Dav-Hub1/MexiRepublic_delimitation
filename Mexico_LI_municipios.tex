\documentclass[floatfix,twocolumn,prd,showpacs,preprintnumbers,amsmath,nofootinbib,amssymb,superscriptaddress]{revtex4-2}

\bibliographystyle{apsrev4-1}


\usepackage[utf8]{inputenc}
\usepackage{graphicx}
% \usepackage{array,eqnarray}


\newcommand{\stackss}[2]{{\begin{subarray}{l}{\textrm{\tiny #1}}\\[-0.3ex] \textrm{\tiny #2}\end{subarray}}} %stacked superscript

\usepackage{hyperref}
\hypersetup{colorlinks,citecolor=blue,urlcolor=blue,linkcolor=black}
\usepackage{float}
\usepackage[nameinlink,capitalize]{cleveref}
\usepackage{tcolorbox}

\renewcommand{\equationautorefname}{Eq.}
\renewcommand{\figureautorefname}{Fig.}
\renewcommand{\sectionautorefname}{Sec.}
\crefname{section}{Sec.}{Secs.}
\crefname{equation}{Eq.}{Eqs.}
\def\ccite#1{Ref.~\cite{#1}} % cite with prefix
\def\ccites#1{Refs.~\cite{#1}} % cite with prefix plural

\begin{document}

\keywords{
flow
}

\title{
heavy quarks
}

\author{David de la Cruz}
\affiliation{CS\\
Mex}

% Custom definitions and shortcuts

\def\d{\mathrm{d}}

\def\luisnote#1{\textcolor{red}{\textbf{Luis:  #1}}}

% scales
\def\muphys{\bar{\mu}_B} 
\def\mubar{\bar{\mu}} %

% correlators
\def\GE{G_E}
\def\GB{G_B} %
 
% coefficients
\def\Zmatch{Z_{\mathrm{match}}} %

\begin{abstract}
We analyze the color-magnetic (or ``$B$") field 

 to $\overline{\text{MS}}$, are used to resolve nontrivial renormalization issues. 

\end{abstract}

\maketitle


\section{Introduction}

Hi


\section{Renormalization}
\label{sec:renorm}

In a completely analogous way one can also define $\GE$, the correlation function for electric fields,
\begin{equation}
\label{GEdef}
    \GE(\tau) = -L^{-1} \langle \Tr G_{01}(0) U(0,\tau) G_{01}(\tau) U(\tau,\beta) \rangle,
\end{equation}
where the $-$ sign accounts for two factors of $i$ in the continuation of Minkowski to Euclidean electrical fields.%
\footnote{$\GE(\tau)$ defined in this way is positive because the Euclidean electric field is time-reflection odd.}
The two spectral functions $\rho_E$ and $\rho_B$ then determine the momentum-diffusion coefficient $\kappa$ via
\begin{equation}
\label{kappa}
    \kappa = \lim_{\omega \to 0} \frac{2T}{\omega}
    \left( \rho_E + \frac{2}{3} \langle \mathbf{v}^2 \rangle \rho_B \right),
\end{equation}
where $\langle \mathbf{v}^2 \rangle$ is the mean-squared velocity of the heavy quark. The velocity, as determined by thermodynamics, depends on the ratio $T/M$, which is the way the dependence on the thermal heavy quark mass $M$ enters.
The operators in \cref{Geucl}, \cref{Gmink} require regularization and their regularized values will generically depend on the scale or procedure used.
For the case of electric field correlators on a Wilson line this regularization turns out to be harmless;
the operator determining $\GE$ turns out to be finite and regulator-independent within $\MSBAR$ and gradient flow regulators, and its value in these two schemes is equal.
But the issue is more complicated for $G_B(\tau)$, as we review below.

\section{Lattice details}
\label{sec:latt}



\begin{table}[t]       
    \centering
    \begin{tabular}{ccrccccc}                            
    \hline \hline
    $a$ (fm) & $a^{-1}$ (GeV) & $N_{\sigma}$ & $N_{\tau}$ & $\beta$ & $T/T_{c}$ & No. configuration\tabularnewline
    \hline
    0.0178 & 11.11 & 96 & 96  & 7.1920 &  0.3712  & 1000 \tabularnewline
    0.0140 & 14.14 & 96 & 120  & 7.3940 &  0.3780  & 1000 \tabularnewline
    0.0117 & 16.88 & 96 & 144  & 7.5440 &  0.3761  & 1000 \tabularnewline
    \hline \hline
    \end{tabular}
    \caption{ The lattices at $T<T_c$ (``zero temperature'') for determining the gauge coupling.
    }
    \label{tab_zeroT}
\end{table}

\section{Determination of gauge coupling and matching factor}\label{sec:coupling}

\begin{align}
    \label{harlander}
    \alphaflow & = \alphamsbar
    \left( 1 + k_1 \alphamsbar + k_2 \alphamsbar^2 \right)
    \quad \mbox{for} \; \mubar^2  = \muflow^2 \,,
    \\
    k_1 & = 1.098 + 0.008 N_f \,, \nonumber \\
    k_2 & = -0.982 - 0.070 N_f + 0.002 N_f^2 \,. \nonumber
\end{align}


\begin{figure*}
    \null \hfill
    \includegraphics[scale=1]{BB_cont_quality_relflow_025.pdf}
    \hspace{2.1cm}
    \includegraphics[scale=1]{BB_cont_quality_relflow_030.pdf}
    \hfill \null
    \caption{Bare color-magnetic correlator $G_B$, tree-level-improved and normalized to its free counterpart $G^\text{norm}$, as a function of squared lattice spacing $a^2$ (or equivalently $1/N_\tau^2$ at fixed temperature $T=1/(a N_\tau)$) at the smallest (left) and largest (right) flow time in units of $\sqrt{8\tauf}/\tau$ according to \cref{eq:flow-extr-window}. The dashed lines and data points at $1/N_\tau^2=0$ represent the linear-in-$a^2$ continuum extrapolation. Statistical errors are smaller than the error bar linewidth.}
\label{fig:corr-cont-extrapo}
\end{figure*}

\section{Correlator computation}
\label{sec:extrap}

Hi

\subsection{Extrapolation}

Hi

\subsection{Correlator}

Hi

\section{Reconstruction}
\label{sec:continue}

Hi

\section{Results and conclusions}
\label{sec:results}

Hi

\section*{Acknowledgements}

Hi


\bibliography{refs}


\end{document}
